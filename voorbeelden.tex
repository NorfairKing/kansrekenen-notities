\documentclass[main.tex]{subfiles}
\begin{document}

\chapter{Voorbeelden}
\label{cha:voorbeelden}



\section{Combinatoriek}
\subsection*{Variaties}
\begin{vb}
  Het aantal manieren om $4$ studenten uit $10$ aan te duiden om $4$ verschillende oefeningen te maken is $V_{10}^{4}$.
\end{vb}

\subsection*{Herhalingsvariaties}
\begin{vb}
  Het aantal verschillende bytes is $\overline{V}_{8}^{2}$.
\end{vb}

\subsection*{Permutaties}
\begin{vb}
  Het aantal manieren om $5$ personen aan een ronde tafel te zetten is $P_{5}$.
\end{vb}

\subsection*{Herhalingspermutaties}
\extra{voorbeeld}

\subsection*{Combinaties}
\begin{vb}  
  Het aantal manieren om een groepje van $4$ studenten uit $10$ aan te duiden is $C^{4}_{10}$.
\end{vb}

\begin{vb}
  Het aantal manieren om twee teams van $6$ uit $12$ spelers te kiezen is $\nicefrac{C^{6}_{12}}{2}$.
\end{vb}

\begin{vb}
  Het aantal manieren om $5$ keer hetzelfde aantal ogen te gooien met $5$ dobbelstenen is $C_{6}^{1}$.
\end{vb}

\begin{vb}
  Het aantal manieren om $4$ keer hetzelfde aantal ogen te gooien met $5$ dobbelstenen is $C_{6}^{1}\cdot C_{5}^{1}$.
\end{vb}

\subsection*{Herhalingscombinaties}
\extra{voorbeeld}

\newpage 
\section{Kansruimten}
\extra{voorbeelden van sigma-algebra}




\end{document}

%%% Local Variables:
%%% mode: latex
%%% TeX-master: t
%%% End:
